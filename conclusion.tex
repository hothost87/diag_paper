\section{Conclusions}
\label{concl}

Discovering the causes of a static analysis being impractical is critical for developing novel automated software tools. In this paper, we have designed the first systematic support for this time-consuming process, presenting automatic root-cause localization and improvement suggestion. To diagnose a JavaScript points-to analysis, we instrument the propagation system with labels that keep track of this history of points-to propagation. This information is then used to identify the pointers whose imprecision have large impact on the overall analysis performance and/or precision. The improvement suggestion algorithm takes the dynamic information to simulate the benefits of various context-sensitive analyses to improve the precision on the root causes via context sensitivity. We have performed evaluation on two sets of benchmarks. The results on library applications suggest that the localization algorithm is capable of identifying a small set of root causes that significantly affect the analysis performance and precision. Applying context sensitivity specifically on these root cause functions resolves the scalability problems that occur on both the imprecise analysis and the whole-program combined context-sensitive analysis. The analysis that automatically applies the suggested context sensitivity on the root cause functions achieves a better balance between precision and performance for most programs in Benchmarks II, demonstrating the effectiveness of the improvement suggestion algorithm.

In the future, we would like to improve the usability of our approach with an interactive user interface; therefore, it becomes more useful for both analysis designer and end user for developing new program analysis techniques and building software tools. We plan to further investigate the currently preliminary improvement suggestion for including other advanced analysis techniques (e.g., program transformation) in the recommendations.